\chapter{Related Work}



\section{Smart Home Security and Privacy}
The mass adoption of IoT devices in both enterprises and smart environments have increased the security issues. According to Alferidah \cite{Iotissues} and Swessi \cite{iotissues1} this issues is presented in all layers of the IoT systems, hardware, software as well as communication. The nature of information sharing also introduces privacy issues in smart environments \cite{Iotissues}. Alferidah \cite{Iotissues} and Swessi \cite{iotissues1} has created a overview of all the most critical vulnerabilities and possible counter measures against the attacks. 

Smart environment features is able to trigger events based on other information from other IoT devices without user interaction. Gu \cite{eavsIoT} did a research on mining of wireless Zigbee traffic in a smart office environment. They were able to identify and attribute 35 different events only by eavesdropping the wireless traffic. By analysis these events the research identified that they could get private information about the routines and actions from the people working the office. 

\section{Robot Vacuum Cleaner Security and Privacy}
The popularity of robot vacuum cleaners raises the concern for information security and privacy issues.   \cite{Roborockvulnerability} looked at the security implementation and vulnerabilities on a Roborock S7. They discovered that the security on the robot vacuum cleaner was reasonably secure, with closed unused ports, encrypted traffic between the vacuum and other smart devices. Due to ethical concerns, no test were done towards the could services handling the sensitive data. During the setup stage they discovered that all devices within the IEEE 802.11 Wi-Fi coverage of the vacuum cleaner could connect to the device to configure the initial configuration, regardless if the device supports the application or not. According to \cite{Roborockvulnerability} the Roborock S7 is vulnerable against DHCP starvation attack from rouge devices on the same network. The authors suggested that networks used to control a Roborock should at least have basic authentication to avoid rouge devices. 
A similar research is done in \cite{Neato} where the selected robot vacuum cleaner was produced by Neato. In this research the authors evaluated both the physical robot as well as communication and security towards the could service and application. They discovered that week cryptography and shared private crypto key among their devices resulted in a huge privacy risk. The collected data reviled personal information about the customers everyday life such as routines, apartment size, pets, number of residence. 

\cite{lindaeavesdropping} did a research paper on eavesdropping private information with a laser sensor on a robot vacuum cleaner and extracting the information through a side-channel. Through the research they were able to sense vibrations in objects like pager bags on the floor and detect words said by humans. By sensing vibrations on objects from television or music speakers they were able to identify songs and tv shows, with high precision. \cite{lindaeavesdropping} suggested that manufactures would have to make security implementations so that user faced data is not include the high resolution which enables attacker to extract private information.

Nguyen \cite{robotvacuum_voulne_nguyendeep}, Kaminski \cite{robotvacuum_voulne1_kaminski2016averting} and Torgilsman \cite{robotvacuum_voulne2_torgilsman2020ethical} all address security and privacy concerns with the deployment off different robot vacuum cleaners. In the researches \cite{robotvacuum_voulne_nguyendeep} and \cite{robotvacuum_voulne2_torgilsman2020ethical} used STRIDE treat analysis framework to identify and categorize the different vulnerabilities. They actively attack the robot vacuum cleaner to expose the information. They both found several security and privacy issues related to setup, LAN and cloud communication for these vacuum cleaners. All of them proposed security improvements that should be implemented by the vendors to keep the devices safer to use. 

\section{Eavesdropping and Event Detection}
Researchers in \cite{Eavs_relat_alyami2022wifi} establish a method to capture out-of-network encrypted wi-fi traffic and attribute different IoT devices within a smart home. The research had a 95 percent accuracy of identifying these devices, and in some cases also their working state. Acar \cite{evas_relat_acar2020peek} and his team conducted a similar research on smart environments. They used machin learning to identify devices and their action, these devices used wi-fi, zigbee and Bluetooth. They also suggested counter measurements that can be implemented to defend against passive eavesdropping attribution.
Xiong \cite{evas_relat_xiong2022network} proposes a network traffic flow mechanism to limit the possibilities to attribute IoT devices and events. Is inject dummy traffic, and delay random traffic packets to mix the transmitting sequence. This defence mechanism creates more delay and latency. 
Trimananda \cite{pingpong_trimananda2020packet} have created a tool to learn and create rules IoT devices, based on wifi and wan traffic. The traffic in the research is encrypted, and they have only used packet lengths, and source and destination address. The tool can trigger events and capture traffic, event data sequences are then used to i identify different events. They also include detection of a robot vacuum cleaner, but only for one common action. 