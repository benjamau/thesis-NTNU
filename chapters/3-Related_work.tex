\chapter{Related Work}
In this chapter work related to the topics; \textit{Smart home security and privacy, robot vacuum cleaner security and privacy} and \textit{eavesdropping and event detection} will be presented in different subsections.


\section{Smart Home Security and Privacy}
The mass adoption of IoT devices in smart environments have increased the security issues within these environments. According to Alferidah and Jhanjhi\cite{Iotissues}, and Swessi and  Idoudi \cite{iotissues1} this issues is presented in all layers of the IoT systems, hardware, software as well as communication. The nature of information sharing also introduces privacy issues in smart environments \cite{Iotissues}. Alferidah and Jhanjhi \cite{Iotissues} has created an overview of the most critical vulnerabilities and possible counter measures. 

IoT smart integration, enables controllers to trigger actions based on sensor data, without user interaction. Gu et al. \cite{eavsIoT} did a research on wireless Zigbee traffic mining, in a smart office environment. They were able to identify and attribute 35 different events only by eavesdropping the wireless traffic. With traffic analysis and event detection, they were able to expose private information about the office routines.  

\section{Robot Vacuum Cleaner Security and Privacy}
The popularity of robot vacuum cleaners raises the concern for information security and privacy issues. Sundström and Nilsson \cite{Roborockvulnerability} looked at the security implementation and vulnerabilities on a Roborock S7. They discovered that the security on the robot vacuum cleaner was reasonably secure. Due to ethical concerns, the could service security was not in scope. During the setup stage they discovered that all devices within wireless coverage of the vacuum cleaner, could add initial configuration, regardless of application support. According to Sundström and Nilsson \cite{Roborockvulnerability} the Roborock S7 is vulnerable against DHCP starvation attack from rouge devices on the same network. The authors suggested that networks used to control a Roborock should at least have basic authentication requirements, to avoid rouge devices. 
A similar research is done by Ullrich et al.\cite{Neato} where the robot vacuum cleaner was produced by Neato. In this research the authors evaluated both the physical robot as well as communication and security towards the could service and application. They discovered that week cryptography and shared private password among their devices resulted in a huge privacy risk. The collected data reviled personal information about the customer's routines, apartment size, pets and number of residence. 

Sami et al. \cite{lindaeavesdropping} did a research paper on private information eavesdropping, based on laser sensor data for a robot vacuum cleaner.  This sensor data was execrated through a side-channel on the targeted robot vacuum cleaner. Through the research they were able to sense vibrations in objects like pager bags and detect words said by humans in the environment. By sensing vibrations on objects from television or music speakers they were able to identify songs and tv shows, with high precision. They suggested that manufactures would have to make security implementations, limiting high precision data to be extracted.

Nguyen \cite{robotvacuum_voulne_nguyendeep}, Kaminski et al. \cite{robotvacuum_voulne1_kaminski2016averting} and Torgilsman and Bröndum\cite{robotvacuum_voulne2_torgilsman2020ethical} all address security and privacy concerns with the deployment off different robot vacuum cleaners. They used the STRIDE treat analysis framework, to identify and categorize the different vulnerabilities. They executed attacks towards the robot vacuum cleaner to expose information. They found several security and privacy issues related to setup, LAN and cloud communication for these vacuum cleaners. All of them proposed security improvements that should be implemented by the vendors. 

\section{Eavesdropping and Event Detection}
Alyami et al. \cite{Eavs_relat_alyami2022wifi} establish a method to capture out-of-network encrypted wi-fi traffic, and attribute different IoT devices within a smart environment. The research had a 95 percent accuracy of identifying these devices, and in some cases also their working state. Acar et al. \cite{evas_relat_acar2020peek} also conducted a similar research on smart environments. They used machin learning to identify devices and their action. These devices used wi-fi, zigbee and Bluetooth. They also suggested counter measurements that can be implemented to defend against passive eavesdropping attribution.
Xiong et al. \cite{evas_relat_xiong2022network} proposes a network traffic flow mechanism, to limit the possibilities to attribute IoT devices and events with eavesdropping. They inject dummy traffic, and delay random traffic packets to mix the network traffic sequence. This defence mechanism creates more delay and latency within the environment, and can disrupt some devices and functionalities.

Trimananda et al. \cite{pingpong_trimananda2020packet} have created a tool to learn and create detection rules for IoT devices, based on wifi and wan traffic. The traffic in the research is encrypted, and they have only used packet lengths, and IP address as attributes. Events are triggered through the tool, and corresponding traffic capturing is initiated. Time stamp and event capturing files are then used to train a ML algorithm to create event signatures.


