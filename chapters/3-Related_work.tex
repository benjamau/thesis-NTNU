\chapter{Related Work}
This chapter introduce existing research and literature relevant for this thesis' topic. Presented work are focused around security and privacy issues, related to the topics: \gls{IoT}, smart environment and robot vacuum cleaners. Further, related work on different eavesdropping attacks and possible countermeasures, as well as IoT event detection are described.

\section{Smart Home Security and Privacy}
The use of \gls{IoT} devices in smart environments have increased the security issues within these environments. According to Alferidah and Jhanjhi \cite{Iotissues}, and Swessi and  Idoudi \cite{iotissues1} these issues are presented in all layers of the \gls{IoT} systems hardware, software and communication. The nature of information sharing also introduces privacy issues in smart environments \cite{Iotissues}. Alferidah and Jhanjhi \cite{Iotissues} have created an overview of the most critical vulnerabilities and possible counter measures in an \gls{IoT} environment. 

\gls{IoT} smart integration enables controllers to trigger actions based on sensor data, without user interaction. Gu et al. \cite{eavsIoT} did a research on wireless Zigbee traffic mining in a smart office environment. They were able to identify and attribute 35 different events only by passively eavesdropping the wireless traffic. With further analysis they were able to expose private information about the office routines based on this traffic.  

\section{Security and Privacy Challenges of Robot Vacuum Cleaners}
The popularity of robot vacuum cleaners raises the concern for information security and privacy issues. Sundström and Nilsson \cite{Roborockvulnerability} looked at the security implementation and vulnerabilities on a Roborock S7. They discovered that the robot vacuum cleaner was reasonably secure. Due to ethical concerns, the cloud service security was not in scope. During the setup stage they discovered that all devices within wireless coverage of the vacuum cleaner, could add initial configuration, regardless of application support. According to Sundström and Nilsson \cite{Roborockvulnerability} the Roborock S7 was vulnerable against \gls{DHCP} starvation attack from rouge devices on the same network. The authors suggested that networks used to control a Roborock should at least have basic authentication requirements, to avoid rouge devices. 
A similar research is done by Ullrich et al. \cite{Neato} where the robot vacuum cleaner was produced by Neato. In this research the authors evaluated communication and security towards the cloud service and application. They discovered that week cryptography and shared private keys among the devices resulted in a huge privacy risk. The collected data revealed personal information about the customers routines, apartment size, pets and number of residents. 

Sami et al. \cite{lindaeavesdropping} did a research on private information eavesdropping, based on laser sensor data of a robot vacuum cleaner.  This sensor data was extracted through a side-channel on the targeted robot vacuum cleaner. Through the research they were able to sense vibrations in objects like pager bags and detect words said by humans in the environment. By sensing vibrations on objects from television or music speakers they were able to identify songs and tv shows, with high precision. They suggested that manufactures have to make security implementations, limiting high precision private data to be extracted from the devices.

Nguyen \cite{robotvacuum_voulne_nguyendeep}, Kaminski et al. \cite{robotvacuum_voulne1_kaminski2016averting} and Torgilsman and Bröndum \cite{robotvacuum_voulne2_torgilsman2020ethical} all address security and privacy concerns with the deployment off different robot vacuum cleaners. They use the STRIDE threat analysis framework to identify and categorize the different vulnerabilities. They executed attacks towards the robot vacuum cleaners to expose information. In addition several security and privacy issues related to setup, \gls{LAN} and cloud communication for these vacuum cleaners was discovered. All of them proposed security improvements that should be implemented by the vendors. 

\section{Eavesdropping and Event Detection}
Alyami et al. \cite{Eavs_relat_alyami2022wifi} establish a method to capture out-of-network encrypted \gls{Wi-Fi} traffic, and attribute different \gls{IoT} devices within a smart environment. The research had a 95 percent accuracy of identifying these devices, and in some cases also their working state. Acar et al. \cite{evas_relat_acar2020peek} also conducted a similar research on smart environments, using machine learning to identify devices and their actions. These devices used \gls{Wi-Fi}, Zigbee and Bluetooth. They also suggested countermeasurements that can be implemented to defend against passive eavesdropping attribution.
Xiong et al. \cite{evas_relat_xiong2022network} proposes a network traffic flow mechanism to limit the possibilities to attribute \gls{IoT} devices and events with eavesdropping. They inject dummy traffic, and delay random traffic packets to mix the network traffic sequence. This defence mechanism creates more delay and latency within the environment, and disrupted some devices and functionalities.

Trimananda et al. \cite{pingpong_trimananda2020packet} have created a tool to learn and create detection rules for \gls{IoT} devices based on \gls{Wi-Fi} and \gls{WAN} traffic. The traffic in the research is encrypted, and they have only used packet lengths and \gls{IP} address as attributes. Events are triggered through the tool, and corresponding traffic capturing is initiated. Timestamp and event capturing files are then used to train a machine learning algorithm to create event signatures. They were able to identify user behaviour within the smart home using this tool. 


