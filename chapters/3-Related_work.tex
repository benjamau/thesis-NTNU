\chapter{Related Work}



\section{Smart Home Security and Privacy}
The mass adoption of IoT devices in both enterprises and smart homes have increased the security issues, according \cite{Iotissues} and \cite{iotissues1} this issues is presented in all layers of the IoT systems, hardware, software as well as communication. The nature of information sharing also introduce privacy issues when used in smart home environments \cite{Iotissues}. \cite{Iotissues} and \cite{iotissues1} has created a overview of all the most critical vulnerabilities and possible counter measures against the attacks. 

Smart home features is able to trigger events based on other information from other IoT devices without user interaction. \cite{eavsIoT} did a research on mining of wireless Zigbee traffic in a smart office environment. They were able to identify and attribute 35 different events only by eavesdropping the wireless, by analysis these events the research identified that they could get private information about the routines and actions from the people in the office. 

\section{Robot Vacuum Cleaner Security and Privacy}
As robot vacuum cleaners have become very popular in modern day smart homes raises the concern for information security and privacy issues related to this. \cite{Roborockvulnerability} looked at the security implementation and vulnerabilities on a Roborock S7. They discovered that the security on the robot vacuum cleaner was reasonably secure, with closed unused ports, encrypted traffic between the vacuum and other smart devices. Due to ethical concerns no test were done towards the could services handling the sensitive data. During the setup stage they discovered that all devices within the IEEE 802.11 Wi-Fi coverage of the vacuum cleaner could connect to the device to configure the initial configuration, regardless if the device supported the application or not. According to \cite{Roborockvulnerability} the Roborock S7 is vulnerable against dynamic host configuration protocol (DHCP) \cite{dhcp} starvation attack from rouge devices on the same network. They suggested that networks which is used to control a Roborock should a t least have basic authentication to avoid rouge devices. 
A similar research is done in \cite{Neato} where the selected robot vacuum cleaner was produced by Neato, in this research they evaluated both the physical robot as well as communication and security towards the could service and application. They discovered that week cryptography and shared private crypto key among their devices resulted in a huge privacy risk. The collected data reviled personal information about the customers everyday life such as routines, apartment size, pets, number of residence. 

\cite{lindaeavesdropping} did a research paper on eavesdropping private information with a laser sensor on a robot vacuum cleaner and extracting the information through a side-channel. Through the research they were able to sense vibrations in objects like pager bags on the floor and detect some words said by people in the room. By sense vibrations on objects from television or music speakers they were able to identify songs and shows played with high precision. \cite{lindaeavesdropping} suggested that manufactures would have to make security implementations so that user faced data is not include the high resolution which enables attacker to extract private information.
