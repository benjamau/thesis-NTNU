\chapter*{Problem description}


\textbf{Title:} Private Information Exposed By The Use Of Robot Vacuum Cleaners In Smart Environments
\newline
\textbf{Student:} Benjamin Andreas Ulsmåg
\newline
\newline

The use of robot vacuum cleaners is rapidly increasing in all kinds of smart environments. Vendors are developing new smart features and APIs to allow integration of third party systems and expand functionality and smart phone applications are used to make it easier for users to personalize their robot vacuum cleaner experience. These applications are delivered through cloud services where commando and control is communicated between the smart environments and cloud services. This communication is generating network traffic in local wireless and cabled networks as well as on the Internet. The traffic generated by the robot vacuum cleaner reflects the actions made by users and can potentially expose user private information if eavesdropped. 

Smart phone applications use encrypted end-to-end communication to mitigate the risk of exposing private information. This kind of security measure is implemented by the application itself and not the network infrastructure. Information about IP-addresses, packet lengths, ports and low level protocols will still be available for attackers carrying out network eavesdropping. The metadata and header information can reveal IoT actions and potentially expose user private information. This thesis aims to address and determine which kind of private information that can be exposed by carrying out passive eavesdropping attack on network traffic generated by a robot vacuum cleaner. 




