\chapter*{Problem description}


\textbf{Title: Private Information Exposed By The Use Of Robot Vacuum Cleaners In Smart Environments}
\newline
\textbf{Student: Benjamin Andreas Ulsmåg}
\newline
\newline

The use of robot vacuum cleaner is rapidly increasing in all kinds of smart environments. Vendors are developing new smart features and APIs to allow maximum functionality and integration flexibility. Vendors develops smart phone applications, to make it easier for users to personalize their robot vacuum cleaner experience. These applications are delivered through cloud services, commando and control is communicated between smart environments and cloud services. This communication is generating network traffic in local wireless, and cabled network as well as Internet traffic. The traffic generated by the robot vacuum cleaner will reflect the actions made by users, and potentially expose user private information. An attacker can eavesdrop the network traffic in different phases in the communication. 

Smart phone application uses encrypted end-to-end communication to mitigate the risk of exposing private information. This kind of security measures are implemented by the application itself and not the network infrastructure. Information about IP-addresses, packet lengths, ports and low level protocol will still be available for attackers conduction network eavesdropping. This metadata and header information can potentially expose user private information. This thesis aims to address and determine which kind of private information that is exposed. 




