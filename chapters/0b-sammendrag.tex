\chapter*{Sammendrag}
Robotstøvsugere er blitt populære IoT enheter og er mye brukt i ulike smarte miljøer. Integrasjon med andre IoT systemer skaper flere sikkerhets og personverns utfordringer ved bruken av disse. Produsenter har utviklet applikasjoner hvor brukere kan konfigurere rengjøring og se informasjon om robotstøvsugeren etter eget ønske. Dette øker integrasjonen mellom brukernes liv og robotstøvsugeren, noe som kan eksponere mer privat informasjon. Industristandarder bruker ende-til-ende kryptering av kommunikasjon mellom applikasjonen, skytjenester og robotstøvsugere for å sikre den private informasjonen som sendes. Selv om denne informasjonen er kryptert, vil metadata i nettverkspakker fortsatt være tilgjengelig gjennom nettverksavlytningsangrep. I dette prosjektet skal vi undersøke hva slags privat informasjon som potensielt kan bli eksponert av denne dataen. En Irobot Roomba i7 ble installert i to forskjellige smarte miljøer hvor et passivt nettverksavlytningsangrep ble gjort mens ulike robotstøvsuger funksjonaliteter ble utført. Analyse av denne dataen avslørte at det var mulig å attribuere flere ulike smarte funksjonaliteter som ble utført av robotstøvsugeren, bare ved å se på Internett trafikken. Ulike signatur-baserte identifiserings algoritmer ble laget og viste en høy deteksjonsrate. Wi-Fi og Internett trafikken til robotstøvsugeren ble sammenlignet og like trafikkmønstre ble funnet, noe som gjør at deteksjonsmetodene også kan brukes for Wi-Fi trafikk. Denne oppgaven tar for seg implementasjon, konfigurasjon og analyse av nettverkstrafikk og presenterer en deteksjonsalgoritme for Irobot Roomba i7 hendelser.