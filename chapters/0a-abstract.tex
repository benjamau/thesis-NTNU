\chapter*{Abstract}

Robot vacuum cleaners are widely used IoT devices, and are deployed in all kinds of smart environments. Mass abduction and integration with IoT systems, introduce more security and privacy issues related to the operation of these devices. Vendors have developed smart phone application were users can personalize cleaning and view information about the vacuum cleaner or discovered environments. This increase the integration between user's life and the robot vacuum cleaner, which potentially exposes more private information. Industry standards include end-to-end encryption between application, could service and robot vacuum cleaner, to secure the private information transferred. Regardless of encryption, network header metadata, is still available through network eavesdropping attacks. Through this project we investigated the potential private information exposed by this metadata. We deployed an Irobot Roomba i7 in two different smart environments, simulation real life operations. Within the two environments we captured WiFi and wide area network (WAN) traffic, while triggering a series of different smart features. Analysis reviled that it was possible to attribute different events triggered on the Irobot Roomba i7, only based on metadata in the WAN traffic capture. Different signature based detection algorithms are proposed, with a high detection rate. WiFi and WAN capturing metadata was compared, and similar patterns were identified, making the detection method applicable for WiFi eavesdropping as well. This thesis covers the implementation, capturing and analysis of this traffic, the results are summarized and used to propose event detection algorithms.    


