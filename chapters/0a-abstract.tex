\chapter*{Abstract}

Robot vacuum cleaners are popular IoT devices and are deployed in all kinds of smart environments. Mass abduction and integration with IoT systems introduce more security and privacy issues related to the operation of these devices. Vendors have developed smart phone applications where users can personalize cleaning or view information about the vacuum cleaner. This increase the integration between user's life and the robot vacuum cleaner, which potentially exposes private information. Industry standards include end-to-end encryption between the application, cloud service and robot vacuum cleaner to secure the private information exchanged. Regardless of encryption, network header metadata is still available through network eavesdropping attacks. In this project we investigated the potential private information exposed by this metadata. An Irobot Roomba i7 was deployed in two different smart environments where passive network eavesdropping was conducted during smart feature triggering. Analysis revealed that it was possible to attribute different events triggered on the Irobot Roomba i7, only based on metadata in the WAN traffic capture. Different signature-based detection algorithms are proposed, with a high detection rate. Wi-Fi and WAN capturing metadata were compared and similar patterns were identified, making the detection method applicable for Wi-Fi eavesdropping as well. This thesis covers the implementation, capturing and analysis of network traffic and proposes event detection algorithms.    


