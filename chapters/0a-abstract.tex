\chapter*{Abstract}
Robot vacuum cleaner has become a popular IoT devices to install in smart environments. Users can configure them to clean where ever suits them. Integration towards other IoT systems are also developed, and the robot vacuum cleaner can start to clean when you lock you door, close you garage or leave you home. Users habits and routines are often closely related to when the robot vacuum cleaner are running, this introduce privacy challenges. Researchers have investigated which private information exposed for smart environments, by correlating events from light bulbs, door locks, motion sensors and robot vacuum cleaner. Other have looked into security issues om the robot vacuum cleaner itself. This thesis address issue of exposed private information only based on robot vacuum cleaner event attribution. An Irobot Roomba i7 robot vacuum cleaner was deployed in two different smart environments. Six different events, \textit{Automated cleaning, scheduled cleaning, application triggered cleaning, physical triggered cleaning, application start} and \textit{bin remove} was chosen, and triggered a series of times in these two smart home environments. Captures of WLAN and WAN network traffic was captures for all the six events. This data was processed and analysed, we were able to identify signatures for five events and propose a detection algorithm. The proposed event detection algorithm was evaluated on event captures in three new live smart home environments. Four of five events was attributed with 100\% accuracy. The thesis propose a methodology, signatures and detection algorithm to attribute Irobot Roomba i7 events with high confidence, based on network traffic captures.  
