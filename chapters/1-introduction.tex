\chapter{Introduction}

 This chapter will start by briefly introduce the problem domain and motivation for this master thesis. Further it will address the research objectives, specify the scope and limitations, and summarize the scientific contribution.  
 
\section{Problem Domain}

The increase of Internet of Things (IoT) and deployment of smart environment are rapidly increasing, and are expected to increase further \cite{iotgrowth}. IoT 
devices are designed to automate and streamline users daily activities and duties. Robot vacuum cleaners, smart lighting, smart garage ports and smart door locks, are becoming a part of every smart home environment. The close integration between IoT devices and users lifes, introduces new security and privacy challenges.

Robot vacuum cleaners has become a popular smart environment device. These robots can automate floor cleaning, based on users preferences and customization \cite{roboticvacuumcleaner2021}. Integration with other IoT devices allows cleaning to be triggered based human action, which will potentially expose users behavior and routines. 

Other researchers have address security challenges on robot vacuum cleaners with penetration testing, vulnerability assessments and active network eavesdropping and interception. There has also been conducted research about passive eavesdropping in smart home environments. These have only identified action on the robot vacuum cleaners and not attributed the different events triggered. Robot vacuum cleaner event attribution, and privacy challenges associated with this, is not addressed.

\section{Research Objectives}
%what goals would you like to achieve? Please clearly define your project goals. 
%What are your research questions?

The goal of this thesis is to identify private information exposed in a smart environment, only based network traffic generated by a robot vacuum cleaner. We want to address this from an attackers perspective, and only use passive eavesdropping in the different phases of network communication. To be able to extract user private information, we want analyze and identify traffic pattern signatures. Further we want to use these signatures to attribute different events within a smart environment. The thesis' three research questions are listed below. 

\begin{enumerate}
    \item Which private information can be gathered from robot vacuum cleaner by carrying out a passive eavesdropping attack in a smart environment?
    \item How can the information exposed by the eavesdropping attack be misused by an attacker?
    \item Which security measures can be implemented to limit the exposed data and decrease the risk of misuse?
\end{enumerate}

\section{Scope and Delimitation}

The scope of this thesis is passive eavesdropping of wireless local area network (WLAN) and wide area network (WAN) traffic. No actions that will effect the traffic flow will be included. This excludes, traffic shaping, man-in-the-middle-attacks, traffic injection and similar actions. All traffic capturing and analysis, is from the perspective of an attacker. Only information that is available in the capturing files are therefor included in the thesis' analysis. This excludes decryption of traffic or knowledge about other local configurations and passwords within the environment or devices. 

Irobot Roomba i7 is the only robot vacuum cleaner considered in this thesis. This robot vacuum cleaner is connected to a separate WLAN during the entire data capturing process, allowing only cloud based communication. Local IoT communication and influence is therefore not included. Environment and network infrastructure is delimited to only basic Internet access. This exclude security implementations of firewalls, access-lists, identity management and multicast addressing, which could effect the communication. 

The complexity of eavesdropping is also not included, due to the large variety of solution in different smart environments. WAN interfaces are delivered by Internet service providers (ISP). Access to this traffic flow will not be considered, and a simulated WAN Interface is created within the local area network (LAN) of the environments. 

Analysis done through Wireshark and basic python scripting. Signature is therefore identified only by human manual analysis through these tools. This limits the analysis to only look at overall characteristics or initial traffic and Machine learning (ML).

\section{Contribution}
%Briefly describe the contribution of your work.
This thesis propose a method to extract network signatures for robot vacuum cleaners based on LAN, WLAN or WAN traffic. It includes a detection algorithm and signatures which can identify four different events, and one event collection on an Irobot Roomba i7, only based in the encrypted WAN traffic. The evaluation propose potential privacy challenges based on the results, as well as security defense mechanisms which could mitigate the impact and risk of passive eavesdropping attacks. 

\section{Thesis Structure}
%Briefly describe how the rest of your thesis is structured
This thesis is divvied into the chapters \textit{Background, Related work, Method, Analysis and Results, Evaluation, Discussion and Conclusion}. First the Background chapter will present relevant information needed to understand the topic. Related work will cover existing research on this area. Next the Method will present the different processes of selection, configuration, processing and analysis. Further the analysis and results will be presented. This is followed by an evaluation chapter to evaluate the research results. Last a discussion and conclusion chapter will compile the thesis' findings. 



