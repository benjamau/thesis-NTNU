\chapter{Introduction}

 This chapter will briefly introduce the problem domain and motivation for this master thesis. It will also address the research objectives and questions, specify the scope and limitations, and summarize the contribution.   
\section{Problem Domain}
%Clearly introduce the problem domain of your research 
%What problem(s) you attempt to address? What is your motivation?
%How did other researches solve the problem(s) Are there still some gaps
The increase of Internet of Things (IoT) devices connected and deployed in smart environment is rapidly increasing, and is predicted to expand more in the future \cite{iotgrowth}. These devices are designed to help humans with daily tasks, and can be everything from door locks, TVs, vacuum cleaners, cameras and cars. As these devices are integrated into our lifes, they also introduce security and privacy issues. Robot vacuum cleaners are popular IoT devices to deploy in smart environments such as homes and business buildings, to automate the cleaning of floors \cite{roboticvacuumcleaner2021}. The motivation for this research is to investigate if usage of such devices will expose user sensitive information or not. 

\section{Research Objectives}
%what goals would you like to achieve? Please clearly define your project goals. 
%What are your research questions?

The goal of this research is to identify if a robot vacuum cleaner by itself, exposes private information or not. In additional to this, identify the level of detection which is possible to perform, with basic human analysis of the smart environment traffic. Research questions for this master thesis is listed below.
\begin{itemize}
    \item Which private information can be gathered from robot vacuum cleaner by carrying out a passive sniffing attack in a smart environment?
    \item How can the information exposed by the sniffing attack be misused by an attacker?
    \item Which security measures can be implemented to limit the exposed data and decrease the risk of misuse?
\end{itemize}

\section{Scope and Delimitation}
%In this subsection, you should clearly state the scope of your work. What will be included?
%What is beyond you project's consideration?
This research will only include traffic capturing and analysis of traffic generated by an Irobot roomba i7, robot vacuum cleaner. There is no decryption of any data performed, all data is in raw capture, and sorted by Wireshark based on plain text data. The environments used to perform the tests are different, but the same infrastructure is used. Local smart environment differences such as firewalls or other security mechanisms is not included. WAN simulation is done by local configuration by the researchers, ISP added features is not considered. 

The complexity and methods to carry out network eavesdropping attacks on an ISP WAN interface or infrastructure is not considered in this research.
\section{Contribution}
%Briefly describe the contribution of your work. 
This research identifies the level of work that an attack will have to do to expose private information about users of Robot vacuum cleaners. It presents signatures and algorithms which can be used to identify different events executed on a Irobot Roomba i7 robot vacuum cleaner. Information and discussion on how the identification could be more precise and how it can be automated is also presented. 

\section{Thesis Structure}
%Briefly describe how the rest of your thesis is structured
This subsection will present the structure of the thesis. First, a chapter to present relevant background information, to ensure that readers will have the correct background knowledge about the topic. Next the methodology will be presented in a separate chapter. Here selection of devices and environments, as well as how the events and analysis is executed is presented. This will be followed up by a chapter presenting the analysis and result of the research. Next chapter will then be an evaluation chapter, followed by discussion, conclusions and future work.  


