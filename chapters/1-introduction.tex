\chapter{Introduction}
This chapter introduces the problem domain of the master project topic. Research objectives is presented with the associated research questions which the thesis aims to address. Further, the research delimitation and contribution is presented, before the overall structure of the thesis is introduced. 
 
\section{Problem Domain}

The increase of \gls{IoT} and deployment of smart environments are rapidly increasing and are expected to increase further \cite{iotgrowth}. \gls{IoT} 
devices are designed to automate and streamline users daily activities and chores. Robot vacuum cleaners, smart lighting, smart garage ports and smart door locks are becoming a part of every smart home environment. The close integration between \gls{IoT} devices and users lifes, introduces new security and privacy challenges.

Robot vacuum cleaners have become a popular smart environment device. These robots can automate floor cleaning based on users preferences and customization \cite{roboticvacuumcleaner2021}. Integration with other \gls{IoT} devices allows cleaning to be triggered based on human action, which can potentially expose user behavior and routines. 

Other researches have addressed security challenges on robot vacuum cleaners with penetration testing, vulnerability assessments and active network eavesdropping and interception. It has also been conducted research about passive eavesdropping in smart home environments including robot vacuum cleaners, but without detailed analysis of the device. Event attribution and privacy challenges associated with this is therefore not addressed.

\section{Research Objectives}
%what goals would you like to achieve? Please clearly define your project goals. 
%What are your research questions?

The goal of this thesis is to identify private information exposed in a smart environment, only based on network traffic generated by a \gls{RVC}. We want to address this from an attackers perspective, and only use passive eavesdropping in the different phases of network communication. To be able to extract user private information, we analyze network traffic and attempt to identify traffic pattern signatures. These three research questions were created to address this topic and guide the research.

\begin{enumerate}
    \item Which private information can be gathered from a robot vacuum cleaner by carrying out a passive network eavesdropping attack in a smart environment?
    \item How can the information exposed by the eavesdropping attack be misused by an attacker?
    \item Which security measures can be implemented to limit the exposed data and decrease the risk of misuse?
\end{enumerate}

\section{Scope and Delimitation}

The scope of this thesis is passive eavesdropping of \gls{WLAN} and \gls{WAN} traffic, this excludes actions that will effect the traffic such as traffic shaping, man-in-the-middle-attacks, traffic injection and similar actions. All traffic capturing and analysis are from the perspective of an attacker. Only information that is available in the capturing files are therefore included in the thesis' analysis. This excludes decryption of traffic or knowledge about other local configurations and passwords within the environments or devices. 

Irobot Roomba i7 is the only robot vacuum cleaner considered in this thesis. This robot vacuum cleaner is connected to a separate \gls{WLAN} during the entire data capturing process, allowing only cloud based communication. Local \gls{IoT} communication and influence is therefore not included. Environment and network infrastructure is delimited to only basic Internet access, this excludes security implementations of for example firewalls, access-lists, identity management and multicast addressing, which could affect the communication. 

The complexity of eavesdropping is also not included, due to the large variety of solutions in different smart environments and Internet access. \gls{WAN} interfaces are delivered by \gls{ISP}, access to this traffic flow will not be considered and a simulated \gls{WAN} Interface is created within the \gls{LAN} of the environments. 

Analysis is done using Wireshark and basic python scripting. Signature is therefore only identified by human manual analysis through these tools. This limits the analysis to only look at overall characteristics or initial traffic and not machine learning.

\section{Contribution}
This project contributes with research on Irobot Roomba i7 robot vacuum cleaner, including detailed network traffic analysis and successful identification and attribution of different events. Previous researches have addressed the same security and privacy challenges with \gls{IoT} smart environment including robot vacuum cleaners, but focusing on attributing different smart environment events not specifically on a robot vacuum cleaner. Other projects have focused on robot vacuum cleaner, but actively attacked the devices to evaluate the security and privacy issues by exploiting different vulnerabilities detected. This thesis therefore add knowledge about level of event attribution which is possible on an Irobot Roomba i7 and proposes detection and signature for different events. 



\section{Thesis Structure}
%Briefly describe how the rest of your thesis is structured
The rest of this thesis is divided into the chapters \textit{Background, Related work, Method, Analysis and Results, Evaluation, Discussion and Conclusion}. First the Background chapter will present relevant information needed to understand the topic. Related work will cover existing research on this area. The Method will present the different processes of selection, configuration, processing and analysis. Further the analysis and results will be presented. This is followed by an evaluation chapter to evaluate the research results. Lastly a discussion and conclusion chapter will summarize the thesis' challenges, decisions and answer the thesis' research questions. 



