\chapter{Conclusions}
In this thesis, we have proposed event detection algorithms to perform event detection and attribution, of an Irobot Roomba i7 robot vacuum cleaner. Data collection was done in two different smart home environments, where the same robot vacuum cleaner was deployed. A baseline traffic capture was conducted over a time period of 14 days, this capture was used to identify all event irrelevant traffic. Based on these finding all IP-packets less then 98 bytes was excluded for further analysis.  

The result was a reduction of more then 99\% of the captured traffic. The same base-filter was applied to the event capturing files, leaving only the traffic relevant to the actual events. IP-addresees, DNS responses and TCP packet length sequences was analyzed. The findings resulted in event specific signatures and a detection algorithm. Five different signatures was proposed. 

To evaluate the proposed event detection algorithm, we deployed the Irobot Roomba i7 in three new smart home environments. Each of the environments had other IoT devices connected to create a more representable smart home. The detection algorithms detected 4 out of 5 with 100\% accuracy. The last event \textit{Bin Remove} gave False negative in all three evaluation environments.

This thesis' results can with high confidence identify and attribute different events triggered on an Irobot Roomba i7, based on WAN network traffic. This will expose user routines and habits. It is also identified the same type of packet sequence in WLAN, which makes the detection transferable to wi-fi as well. Security measures should therefore be used by the Robot vaccum cleaner vendors ti mitigate the risk of these eavesdropping attacks. 

\section{Future work}
Future research should look into development of an automated tool, which can capture, process and analysis network traffic automatically. This would ease the process, and enable researches to extract similar results from a series of robot vacuum cleaners. Future it could compare different vendors and, evaluate privacy differences cross vendors. This would contribute to better security awareness and design for all users of robot vacuum cleaners. 

Another diction of research would be to look into irobot integration. Create real life smart environments, where devices that can integrate and trigger vacuum cleaner events is deployed. Device attribution and traffic correlation, could expose even more private information about users.   






