\chapter{Conclusions}
The primary objective of this project is to evaluate if private information is exposed by conducting a passive network eavesdropping attack on a smart environment installed with a robot vacuum cleaner. In addition it addresses the potential risks and countermeasures to defend against such attacks.

In order to meet the project requirements, a robot vacuum cleaner survey was conducted to choose the most relevant vacuum cleaner available. The decision was based on popularity and open-source review sites where \textit{Irobot Roomba i7} was identified as the most suitable. In order to determine if private information is exposed, two smart environments were configured to conduct testing and collection of data generated by the robot vacuum cleaner. A series of event objectives were defined based on the potential private information they could reveal if detected. Captures from the different events were analyzed to identify irrelevant traffic to be removed and signatures to be used in event detection. 

To ensure validity of the identified signatures, three evaluation environments were configured presenting as live smart environments. All the events were triggered within these environments and used as input in a signature detection algorithm to determine if the signatures are consistent and possible to detect. 

In conclusion, the identified signatures and detection algorithms were able to identify \textit{Automated cleaning, Application triggered cleaning} and \textit{Application start}, conducted on the Irobot Roomba only based on data from passive network eavesdropping attacks. The thesis also propose different defense mechanisms that would make the proposed signatures and detection to fail. As the time constrains and resources were limited, only one robot vacuum cleaner was used in the project, and the complexity of eavesdropping is not addressed. 

\section{Future Work}
Future research should look into development of an automated tool, which can capture, process and analysis network traffic automatically based on the attributes used in this research. This would ease the process and enable researches to extract similar results from a series of robot vacuum cleaners. In addition it could be valuable to compare different vendors and privacy differences cross these vendors. This would contribute to better security awareness and design for all users of robot vacuum cleaners. 

Further analysis of more events and new robot vacuum cleaners would be interesting. Live smart environments can be designed with continuous packet capturing and event triggering based on normal user behaviour. Live detection of such events should also be developed which would decrease the detection time and amount of storage acquired. 






