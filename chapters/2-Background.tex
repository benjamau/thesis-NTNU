\chapter{Background}
In this chapter, we will present fundamental concepts and background of Internet of Things, smart environment and robot vacuum cleaners. Furthermore, we will define and present an overview of traffic eavesdropping and traffic shaping. 

\section{Internet of Things (IoT)}
IoT is a system of interconnected physical and virtual devices communicating and sharing information, using Internet or private networks. Autonomous IoT devices are available for information sharing and event triggering continuously.They can act according to inputs, status or triggers from other IoT devices \cite{atlam2020iot}. IoT systems takes advantage of the large scale information sharing. Intelligence software enables the devices to become smarter and more advanced based on information shared among IoT devices. The devices are heterogeneity and involved numerous different hardware components and software versions and languages \cite{atlam2020iot}. Standardized communication protocols and system architecture, makes the integration between IoT platforms possible. 

Small devices like video cameras, smart door lock ect. have limited local processors and storage. To allow the complexity and price of these IoT devices to have an accepted level for users, could computing is used. Data from the IoT devices are therefore sent to could services, where it is processed. Algorithms will communicate commando and control traffic back to the devices, based on user configuration \cite{pavelic2018internet}. 

The level of local computational power is determined in the latency need to the selected systems. A smart car breaking system needs low latency compared to a robot vacuum cleaner. The smart car system therefore needs more local computational power, these systems are refereed to at edge or fog computing \cite{mocrii2018iot}.  


\subsection{Smart Environments}
Smart environments are identified by their seamless connectivity between the sensors, edge devices and a centralized control system. Data is continuously collected form the smart environment providing the centralized controller with live data \cite{lin2016iot}. This can be analyzed, and trigger actions from the controller to other devices in the environment. In a smart home environments the controller can be notified by the garage port opener or sensor that the car has left the premises. This could trigger a sequence of events such as, lock the door, turn of all lights or start the robot vacuum cleaner. In a smart industry environments the IoT sensors can communicate that temperature or other air quality measurements are off the ideal values. This can then trigger systems to normalize this. Due to availability of this data, users can remotely monitor, automate and control the environments based on their needs and requirements. This can give a personalized user experience and value \cite{mantas2011security}. Several centralized smart home applications are developed to make the user experienced and device integration as easy as possible. Home assistant \cite{homeassistant2023} is an example of this. The application enables integration between IoT systems, based on WEB application programmable interfaces (API). These interactions aims to include as many IoT systems as possible, introducing security and privacy challenges. 

\subsection{Robot Vacuum Cleaners}
Robot vacuum cleaners are popular smart home IoT devices. These robots can clean the smart environment autonomously, and can be configured to clean based on scheduled cleaning tasks, or automatic cleaning based on integration with other IoT systems. Newer models have advanced cleaning and navigation technology, and are able to map their sounding, avoid obstacles and suggest cleaning routines based on season, or the level of dust in the environment. Popular robot vacuum cleaner vendors are Irobot \cite{irobot}, Neato \cite{Neato}, Ecovacs \cite{ecovacs}, Roborock \cite{roborock}. 


\subsubsection{Robot vacuum cleaner communication protocols}
The newest models of all the vednors \cite{irobot}, \cite{Neato}, \cite{ecovacs}, \cite{roborock} and \cite{neatsvor}, uses Wi-fi as their main communication protocol. This is used to communicate with the could service, and presents live data in the associated smart home application.

\paragraph{IEEE 802.11} is a Media access control specification used in modern wi-fi communication \cite{wifi_ieee80211}. Wifi is used to transmit data to the wired smart environment infrastructure, and towards the could services. Traffic can therefor be eavesdropped both during wireless and wired communication \cite{eavesdroppingwifi}. Wi-Fi transmission uses MAC-addresses \cite{macaddress} to determine where the packet origin and destination is. A MAC address includes 48 bits, where the first 24 is used as a organization identifier. The last 24 bits is then used as a unique identifier within an organization. These registers are available online \cite{mac_address_lookup}, and can be used to identify devices connected to wireless or wired networks. 

\section{Traffic eavesdropping}
Traffic eavesdropping is a technique used to collect network traffic, not destiny for the collecting device \cite{eavsdropping_fortinet}. Eavesdropping can be separated into two categories, passive and active. In active eavesdropping an attacker will interfere with the traffic flow. This could be packet injection, modification or disruption. Passive eavesdropping will only collect traffic without any interference. To conduct wireless eavesdropping an attacker will only need to be in wireless coverage of the devices generation the targeted traffic \cite{eavesdroppingwifi}. Wired eavesdropping requires physical or remote access to network devices in the smart environment and is therefore more complex to conduct. 

Wireshark \cite{wireshark}, Tshark \cite{wireshark}, tcpdump \cite{tcpdump} and Microsoft message analyzer \cite{microsoftmessageanalyzer} are some tools that can be used for network eavesdropping. All these tools can monitor traffic received on a specific network interface cards (NIC). During wireless eavesdropping, the wireless NIC needs to be in monitoring mode, and process all packets received and not just those with the correct destination MAC address. For wired eavesdropping an attacker can configure a switch port analyzer (SPAN) port, which duplicated all specified traffic to a selected eavesdropping port. This functionally is used to install intrusion detection systems (IDS) in networks. 

\section{Traffic Shaping}
Traffic shaping is a technique used to shape the network traffic based on policies. This is used to optimize data network, prioritizing traffic and limiting transmission of irrelevant data \cite{traffic_shaping_saeed2017carousel}. This technique is also proposed as a defense mechanism in IoT smart environments. Authors in \cite{traffic_shaping_xiong2022network} proposes a method to shape smart environment traffic to defend against traffic flow analysis attacks. There mechanisms can mitigate risk and increase the complexity of such attacks. 




