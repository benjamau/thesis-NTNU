\chapter{Background}

Internet connected devices is affecting human everyday life and routine more and more. The development in technology enables higher Internet speed, faster computers and huge amount of digital storage increases the usage and price for helpful devices such as air quality monitors, robot vacuum cleaners and smart door locks. All together these connected devices form smart home environment where home equipment can notify and act according to sensors or events triggered by other interconnected devices \cite{atlam2020iot}. A robot vacuum cleaner could clean when the front door is locked and all the smart light bulbs are turned off. For this to be possible the devices will have to use standard protocols and architecture to communicate with each other. 

This section will provide background information about key elements of IoT, the use of these devices to enable a smart home environment. It will also introduce robot vacuum cleaners, common features, specifications and network and application architecture. At the end it will introduce techniques for capturing live traffic in wires and wireless communication. 

\section{Internet of Things (IoT)}
IoT is a collection of interconnected physical and virtual devices communication and sharing information with each other using Internet or private networks. Autonomous devices connected to a network will be available for information sharing, event triggering and actions at any time and can act according to inputs, status or triggers from other inter connected devices \cite{atlam2020iot}. These systems takes advantage of the mass amount of devices to perform large scale information sharing, intelligence software enables the devices to become smarter and more advanced based on the information which is shared among them. The devices are heterogeneity and involved numerous different hardware components and software versions and languages \cite{atlam2020iot}.

The nature of customization of IoT devices makes the computational power and storage different, small devices like video cameras, smart door lock ect. can have limited local processors and storage and more the complexity to more centralized computational power such as could services. Data sensed by the door lock will have to be transferred to the cloud server, processed and replied with the appropriate action. This type of centralized architecture is more salable because the complexity is centralized, the challenge is therefor is ensure secure communication of the information as well as only authorized access to trigger events \cite{pavelic2018internet}. Some IoT systems need low latency and high computational capacity, smart car sensors is an example of this, self break and steering assistant will not have time to establish a secure connection to a could server, transfer information and then receive a action response. This low latency and high capacity introduce the need of edge or fog computing, the computing is therefor added closer to the IoT device if not embedded in the device itself \cite{mocrii2018iot}.  


\subsection{Smart Environments}
Smart environments are identified by their seamless connectivity between the sensors, edge devices and the centralized control system. Data is continuously collected form the smart environment providing the centralized controller with live data. This can be analyzed and, trigger actions from the controller to other devices in the environment. In a smart home environments the controller can be notified by the garage port opener or sensor that the car has left. This could trigger a sequence of events such as, lock the door, turn of all lights or start the robot vacuum cleaner. In a smart industry environments the IoT sensors can communicate that temperature or other air quality measurements are off the ideal values. This can then trigger systems to normalize this. Due to availability of this data, users can remotely monitor, automate and control the environments based on their needs and requirements. This can give a personalized user experience and value. Several centralized smart home applications are developed to make the user experienced and device integration as easy as possible. Home assistant \cite{homeassistans} is an example of this, and makes the environment salable. Many IoT devices are simple and is not developed with the state of the art security and privacy concerns, it is therefor important to take this into count when designing such environments. 

\subsection{Robot Vacuum Cleaners}
Robot vacuum cleaners are popular smart home IoT devices. These robots can clean the smart environment autonomously, and can be configured to clean based on scheduled cleaning tasks, or automatic cleaning based on integration with other IoT smart environment devices. Newer models have advanced cleaning and navigation technology, and are able to map their sounding, avoid obstacles and suggest cleaning routines based on season or the level of dust in the apartment. Popular robot vacuum cleaner vendors are Irobot \cite{irobot}, Neato \cite{Neato}, Ecovacs \cite{ecovacs}, Roborock \cite{roborock} and several others. 


\subsubsection{Robot vacuum cleaner communication protocols}
The newest models of all the vednors \cite{irobot}, \cite{Neato}, \cite{ecovacs}, \cite{roborock} and \cite{neatsvor}, uses Wi-fi as their main communication protocol. This is used to communicate with the could service, which presents live data in the associated smart home application. Some of the vacuum cleaners uses Bluetooth during setup, but not during could service IoT data extraction. Wifi \cite{wifi_ieee80211} is used to transmit data to the wired IEEE 802.11 and IP-network inside the smart environment infrastructure, and towards the could services. Traffic can therefor be eavesdropped both during wireless and wired communication \cite{wiredeavsdropping} \cite{wirelesseavsdropping}.

\paragraph{IEEE 802.11} is a Media access control specification used in modern wi-fi communication, IEEE has is specifying this standard in \cite{wifi_ieee80211}.  Wi-Fi transmission uses MAC-addresses \cite{macaddress} to determine where the packet origin and destination is. These addresses are registered and different manufactures have reserved the 24 bits in a mac address to use on their production. These registers are available online \cite{mac_address_lookup}, this can be used to identify which type of devices are sharing the same shared medium. 

\subsection{Traffic eavesdropping}
Traffic eavesdropping is a technique used to collect network traffic which is not destiny for the collection device \cite{eavsdropping_fortinet}. Fortigate also differs between passive an active eavesdropping. If an attacker only listens into a network transmission without any interference, it is consider a passive attack. The easiest way to execute such attacks are with wireless traffic \cite{eavesdroppingwifi}, but this could also be done in wired IP networks. This is more complex as an attacker will have to gain access to network equipment. 

Wireshark \cite{wireshark}, Tshark \cite{wireshark}, tcpdumo \cite{tcpdump} and Microsoft message analyzer \cite{microsoftmessageanalyzer} are some tools that can be used for eavesdropping on traffic. All these can monitor traffic received on a specific network interface card (NIC), and store the traffic. During wireless eavesdropping, the wireless NIC needs to be in monitoring mode, and process all packets received and not just those with the correct MAC address. For wired eavesdropping an attacker can configure a switch port analyzer (SPAN) port, which duplicated all specified traffic to a selected port. 

\section{Traffic Shaping}





