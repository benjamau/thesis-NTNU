\chapter{Background}
This chapter introduce fundamental concepts and background of Internet of Things and smart environments. It will also present robot vacuum cleaners how they operates and which communication protocols they use. Furthermore, it defines and present an overview of traffic eavesdropping and potential defense mechanisms which is important in regards to the research questions. 

\section{Internet of Things}
\gls{IoT} is a system of interconnected physical and virtual devices communicating and sharing information, using the Internet or private networks. Autonomous \gls{IoT} devices are available for information sharing and event triggering continuously and can act based on inputs, status or triggers from other \gls{IoT} devices \cite{atlam2020iot}. \gls{IoT} systems take advantage of the large scale information sharing. Intelligence software enables the devices to become smarter and more advanced based on information shared among \gls{IoT} devices. The devices includes a number of different hardware components and software versions, while standardized communication protocols and system architecture makes the integration between \gls{IoT} platforms possible \cite{atlam2020iot}. 

Small devices like video cameras, smart door locks or air quality monitors have limited local processors and storage. Complexity and the need for data processing have made vendors integrate their systems to centralized cloud infrastructures. Data from the \gls{IoT} devices is therefore sent to cloud services, where it is processed. Algorithms communicate commando and control traffic back to the devices based on user configuration \cite{pavelic2018internet}. However, the use of cloud introduce latency because sensor data needs to be transferred to the cloud server where it is processed and actions are decided. All this extra transmission latency is not applicable in for example a smart car breaking system because it requires fast decision making. Local computing is therefore distributed closer to the sensor providing low latency decision making, this is referred to as fog or edge computing \cite{mocrii2018iot}.   

\section{Smart Environments}
Smart environments are identified by their seamless connectivity between the sensors, edge devices and a centralized control system. Data is continuously collected from the smart environment providing the centralized controller with live data \cite{lin2016iot}. This data can be analyzed, and trigger actions from the controller to other devices in the environment. In a smart home environment the controller can be notified by a garage port opener or sensor that the car has left and trigger a sequence of events such as locking the door, turning of all the lights or starting the robot vacuum cleaner. In smart industry environments the \gls{IoT} sensors can communicate that temperature or other air quality measurements are outside off the threshold values and then trigger systems to carry out actions to stabilize this. Due to availability of this data, users can remotely monitor, automate and control the environments based on their needs and requirements. This can give a personalized user experience and value \cite{mantas2011security}. Several centralized smart home applications are developed to make the user experienced and device integration as easy as possible, Home assistant \cite{homeassistant2023} is an example of this. The application enables integration between \gls{IoT} systems, based on application programmable interfaces. These interactions aims to include as many \gls{IoT} systems as possible, introducing security and privacy challenges across different \gls{IoT} platforms. 

\section{Robot Vacuum Cleaners}
Robot vacuum cleaners are popular smart home \gls{IoT} devices. These robots can clean the smart environment autonomously, and can be configured to clean based on scheduled cleaning tasks or automatic cleaning based on integration with other \gls{IoT} systems. Newer models have advanced cleaning and navigation technology and are able to map their surroundings, avoid obstacles and suggest cleaning routines based on season or the level of dust in the environment. Popular robot vacuum cleaner vendors are Irobot \cite{irobot}, Neato \cite{Neato}, Ecovacs \cite{ecovacs} and Roborock \cite{roborock}. 


\subsection{Robot Vacuum Cleaner Communication Protocols}
The newest models from all the vendors Irobot \cite{irobot}, Neato \cite{Neato}, Ecovacs\cite{ecovacs}, Roborock\cite{roborock} and Neatsvor \cite{neatsvor}, use \gls{Wi-Fi} as their main communication protocol. \gls{Wi-Fi} is used to communicate with the cloud service, and present live data in the associated smart home application.

\textit{IEEE 802.11} is a media access control specification used in modern \gls{Wi-Fi} communication \cite{wifi_ieee80211}. \gls{Wi-Fi} is used to connect wireless \gls{IoT} devices to the wired smart environment network infrastructure communicating with cloud services. Traffic can therefore be eavesdropped both during wireless and wired communication \cite{eavesdroppingwifi}. \gls{Wi-Fi} transmission uses \gls{MAC}-addresses \cite{macaddress} to determine the packet origin and destination. A \gls{MAC} address includes 48 bits, where the first 24 are used as an organization identifier. The last 24 bits are then used as a unique identifier within an organization \gls{MAC} range. Registers of organizations global \gls{MAC} identifier are available online, and can be used to identify which devices that are connected to wireless or wired networks \cite{mac_address_lookup}. 

\section{Traffic Eavesdropping}
Traffic eavesdropping is a technique used to collect network traffic, not addressed to the collecting device \cite{eavsdropping_fortinet}. Eavesdropping can be separated into two categories, passive and active. In active eavesdropping an attacker will interfere with the traffic flow. This could be packet injection, modification or disruption. Passive eavesdropping will only collect traffic without any interference. To conduct wireless eavesdropping an attacker will only need to be in wireless range of the targeted devices \cite{eavesdroppingwifi} and in wired eavesdropping physical or remote access to network devices in the smart environment is required and increases the complexity. 

Wireshark \cite{wireshark}, Tshark \cite{tshark_filter}, tcpdump \cite{tcpdump} and Microsoft message analyzer \cite{microsoftmessageanalyzer} are some tools that can be used for network eavesdropping. All these tools can monitor traffic received on a specific \gls{NIC}. During wireless eavesdropping the wireless \gls{NIC} needs to be in monitoring mode and process all packets received. For wired eavesdropping an attacker can configure a Switch Port Analyzer (SPAN) port on a \gls{LAN} switch duplicating and forwarding specified traffic to the interface connected to the capturing device. This functionality also have legitimate use cases with implementation of Intrusion Detection Systems in networks as an example. 

\section{Eavesdropping Defense Mechanisms}
Traffic shaping is a technique used to shape the network traffic based on policies. This is used to optimize data networks, prioritizing traffic and limiting transmission of irrelevant data, but also proposed as a defense mechanism in \gls{IoT} smart environments \cite{traffic_shaping_saeed2017carousel}. Authors in \cite{traffic_shaping_xiong2022network} proposes a method to shape smart environment traffic to defend against traffic flow analysis attacks, mitigating the risk and increase the complexity of such attacks. 

Encryption is another popular defense mechanism against network eavesdropping. Traffic can be encrypted in different network layers simultaneously creating multi layer encryption. Applications create end-to-end encryption with for example \gls{TLS} \cite{tls_rfc} or Secure Shell (SSH) \cite{ssh_rfc}. This can also be done on the network layer by using different types of \gls{VPN}s. Two categories are tunneled or transport mode \gls{VPN}s where tunnel mode encrypts the original \gls{IP}-header and add a new, hiding the original source and detonation address \cite{ipsec_rfc} \cite{l2tp_rfc}. Both Internet Protocol Security (IPSec) \cite{ipsec_rfc} and Layer Two Tunneling Protocol \cite{l2tp_rfc} are examples of popular \gls{VPN} protocols. As for \gls{Wi-Fi} communication it has become industry standard to include encryption between the wireless devices and Access Points (AP), authentication can be done with pre-shared keys, certificates or integration with identity access management solution for example Windows active directory or Kerberos. 

The use of global administrated \gls{MAC} addresses introduce privacy and security issues as a \gls{MAC} address can be tracked or identified easily, especially in wireless eavesdropping. Local \gls{MAC} randomization \cite{ietf-madinas-mac-address-randomization-06} can be used to hide the original global \gls{MAC} address associated with an organization and use a new local \gls{MAC} address within a wireless network hiding device information for potential misuse.



